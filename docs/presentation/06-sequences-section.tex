\begin{frame}{Sequences Section}
  \begin{itemize}
    \item Secvențele modelează repetiția structurală din semnal.

    \item Fiecare secvență este un triplet:
      \[
        (\texttt{literal\_length},\ \texttt{match\_length},\ \texttt{offset})
      \]

    \item Fiecare componentă este decodificată printr-un decoder FSE separat.

    \item Prin trei automate finite, care avansează intercalat, este decodificată structura
      reconstrucției.
  \end{itemize}

  \vspace{0.4cm}
  \begin{center}
    \resizebox{0.95\linewidth}{!}{%
    \begin{tikzpicture}[
      font=\sffamily,
      thick,
      >=stealth,
      bitstream/.style={draw=orange!80!black, fill=orange!10, minimum height=3.8cm, minimum width=1.2cm, align=center, rounded corners=2pt},
      fse/.style={draw=purple!80!black, fill=purple!10, minimum height=0.7cm, minimum width=3.4cm, align=center, rounded corners=2pt},
      triplet/.style={draw=green!70!black, fill=green!10, minimum height=0.7cm, minimum width=3.2cm, align=center, rounded corners=2pt}
    ]
      % 1. Bitstream Source
      \node[bitstream] (bit) at (0, 0) {\rotatebox{90}{\textbf{Bitstream}}};

      % 2. Three FSE Automata (Mutate mai la dreapta pentru a face loc)
      \node[fse] (ll) at (5.2, 1.3) {FSE: \texttt{literal\_length}};
      \node[fse] (ml) at (5.2, 0) {FSE: \texttt{match\_length}};
      \node[fse] (of) at (5.2, -1.3) {FSE: \texttt{offset}};

      % Grouping for Automata
      \node[
        draw=purple!80!black,
        dashed,
        inner sep=8pt,
        rounded corners=2pt,
        fit=(ll)(ml)(of)
      ] (fse_group) {};
      \node[above=0.15cm of fse_group.north, font=\sffamily\small, text=purple!80!black]
        {\textbf{Automate Finite Intercalate}};

      % 3. Output Triplet Components (Mutate proporțional mai la dreapta)
      \node[triplet] (t_ll) at (10.6, 1.3) {\texttt{literal\_length}};
      \node[triplet] (t_ml) at (10.6, 0) {\texttt{match\_length}};
      \node[triplet] (t_of) at (10.6, -1.3) {\texttt{offset}};

      % Grouping for Output
      \node[
        draw=green!70!black,
        dashed,
        inner sep=8pt,
        rounded corners=2pt,
        fit=(t_ll)(t_ml)(t_of)
      ] (seq_group) {};
      \node[
        right=0.15cm of seq_group.east,
        font=\sffamily\small,
        text=green!70!black,
        rotate=-90,
        anchor=north
      ] {\textbf{Triplet Secvență}};

      % --- Routing ---

      % Arrows from bitstream
      \draw[->, orange!80!black] (bit.east |- ll.west) -- (ll.west);

      % Săgeata centrală cu textul pus pe două rânduri (align=center)
      \draw[->, orange!80!black]
        (bit.east |- ml.west) -- (ml.west)
        node[
            midway,
            fill=white,
            inner sep=2pt,
            font=\sffamily\scriptsize,
            text=orange!80!black,
            align=center
          ]
          {Consum\\[-0.5ex]Intercalat};

      \draw[->, orange!80!black] (bit.east |- of.west) -- (of.west);

      % Interleaved cyclic state arrows
      \draw[->, purple!80!black, dotted, bend right=45] (ll.west) to (ml.west);
      \draw[->, purple!80!black, dotted, bend right=45] (ml.west) to (of.west);
      \draw[->, purple!80!black, dotted, bend right=60] (of.east) to (ll.east);

      % Arrows to triplet outputs
      \draw[->, purple!80!black] (ll.east) -- (t_ll.west);
      \draw[->, purple!80!black] (ml.east) -- (t_ml.west);
      \draw[->, purple!80!black] (of.east) -- (t_of.west);
    \end{tikzpicture}%
    }
  \end{center}
\end{frame}