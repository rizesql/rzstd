\begin{frame}{Reconstrucția Semnalului în Zstandard}
  \begin{itemize}
    \item Zstandard utilizează un proces cauzal, cu memorie finită (look-back windows).

    \item Semnalul este modelat in două componente:
      \begin{itemize}
        \item \textbf{Literals}: Simboluri emise direct.

        \item \textbf{Sequences}: Referințe la simboluri deja construite (\texttt{literal\_length,
          match\_length, offset})
      \end{itemize}
  \end{itemize}

  \begin{block}{Observație}
    Modelul este echivalent cu o schemă LZ77
  \end{block}

  \begin{center}
    \resizebox{0.75\linewidth}{!}{%
    \begin{tikzpicture}[
      node distance=0pt,
      cell/.style={draw, thick, minimum width=0.8cm, minimum height=0.8cm, align=center, font=\sffamily\Large},
      literal/.style={fill=blue!10, draw=blue!80},
      sequence/.style={fill=green!10, draw=green!70!black}
    ]
      % Array cells
      \node[cell, literal] (N1) {A};
      \node[cell, literal, right=of N1] (N2) {B};
      \node[cell, literal, right=of N2] (N3) {C};
      \node[cell, literal, right=of N3] (N4) {D};

      \node[cell, sequence, right=of N4] (N5) {A};
      \node[cell, sequence, right=of N5] (N6) {B};
      \node[cell, sequence, right=of N6] (N7) {C};
      \node[cell, sequence, right=of N7] (N8) {D};

      \node[cell, literal, right=of N8] (N9) {E};
      \node[cell, literal, right=of N9] (N10) {F};

      % Top Braces (Step Groupings)
      \draw[thick, blue!80, decorate, decoration={brace, amplitude=5pt}]
        (N1.north west) -- (N4.north east)
        node[midway, above=8pt, font=\sffamily\small] {\textbf{Literals}};

      \draw[thick, green!70!black, decorate, decoration={brace, amplitude=5pt}]
        (N5.north west) -- (N8.north east)
        node[midway, above=8pt, font=\sffamily\small] {\textbf{Sequence}};

      \draw[thick, blue!80, decorate, decoration={brace, amplitude=5pt}]
        (N9.north west) -- (N10.north east)
        node[midway, above=8pt, font=\sffamily\small] {\textbf{Literals}};

      % Calculate the horizontal centers of the blocks for perfect arrow alignment
      \path (N1.south) -- (N4.south) coordinate[midway] (LitCenter);
      \path (N5.south) -- (N8.south) coordinate[midway] (SeqCenter);

      % Backward Copy Arrow
      \draw[->, >=stealth, thick, green!70!black]
        (SeqCenter) --
        ++(0,-0.8) -|
        node[pos=0.25, below, font=\sffamily\small]
          {\textbf{Feedback:} \textit{offset} = 4, \textit{match\_length} = 4}
        (LitCenter);

      % Optional dashed lines
      \draw[green!70!black, dashed] (N5.south west) -- ++(0,-0.8);
      \draw[green!70!black, dashed] (N8.south east) -- ++(0,-0.8);
      \draw[blue!80, dashed] (N1.south west) -- ++(0,-0.8);
      \draw[blue!80, dashed] (N4.south east) -- ++(0,-0.8);
    \end{tikzpicture}%
    }
  \end{center}
\end{frame}