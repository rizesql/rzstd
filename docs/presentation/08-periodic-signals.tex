\begin{frame}{Generarea Semnalelor Periodice}
  \small Pentru $\mathtt{offset}< \mathtt{match\_length}$, regiunea din fereastra de look-back
  din care se copiază și fereastra unde se scrie se suprapun, deci o operație de copiere directă
  conduce la acces invalid de memorie.

  \textbf{Soluția rzstd: Generare Exponențială ($\mathcal{O}(\log N)$ copieri)} \\ În loc
  de copiere unitate cu unitate, implementarea dublează iterativ dimensiunea semnalului generat,
  folosind porțiunea deja validată ca bază pentru copia următoare.

  \vspace{0.2cm}
  \begin{center}
    \resizebox{0.9\linewidth}{!}{%
    \begin{tikzpicture}[
      font=\sffamily,
      thick,
      >=stealth,
      seg/.style={draw=blue!80!black, fill=blue!10, minimum height=0.6cm, align=center, rounded corners=2pt, font=\large},
      new/.style={draw=green!70!black, fill=green!10, minimum height=0.6cm, align=center, rounded corners=2pt, font=\large},
      rem/.style={draw=orange!80!black, fill=orange!10, minimum height=0.6cm, align=center, rounded corners=2pt, dashed, font=\large}
    ]
      % Pasul 0
      \node[anchor=east] at (-0.5, 0) {\textbf{Pasul 0:}};
      \node[seg, minimum width=2cm] (p0) at (1, 0) {Perioada ($P$)};
      \node[right=0.2cm of p0, font=\small, text=gray!80!black]
        {Baza inițială (dimensiune = \textit{offset})};

      % Pasul 1
      \node[anchor=east] at (-0.5, -1.1) {\textbf{Pasul 1:}};
      \node[seg, minimum width=2cm] (p1a) at (1, -1.1) {$P$};
      \node[new, minimum width=2cm] (p1b) at (3.2, -1.1) {$P$};

      % Pasul 2
      \node[anchor=east] at (-0.5, -2.2) {\textbf{Pasul 2:}};
      \node[seg, minimum width=4.2cm] (p2a) at (2.1, -2.2) {$P \quad | \quad P$};
      \node[new, minimum width=4.2cm] (p2b) at (6.5, -2.2) {$P \quad | \quad P$};

      % Pasul 3 (Final)
      \node[anchor=east] at (-0.5, -3.3) {\textbf{Pasul k:}};
      \node[seg, minimum width=8.6cm]
        (p3a)
        at
        (4.3, -3.3)
        {$P \quad | \quad P \quad | \quad P \quad | \quad P$};
      \node[rem, minimum width=2.5cm] (p3b) at (10.05, -3.3) {Rest};
      \draw[->, blue!80!black]
        (p3a.south) to[bend right=12]
        node[below=2pt, font=\small] {\textbf{Copiere Trunchiată}}
        (p3b.south);
    \end{tikzpicture}%
    }
  \end{center}
\end{frame}